%\newpage
\subsubsection{\stid{3.13} CLOVER Sub-project SLATE}\label{subsubsect:slate}

\paragraph{Overview}

SLATE (Software for Linear Algebra Targeting Exascale)
provides fundamental dense linear algebra capabilities
to DOE and the HPC community at large.
To this end, SLATE provides
parallel basic linear algebra subprograms (BLAS), norms,
linear systems solvers, least square solvers,
singular value and eigenvalue solvers.

The ultimate objective of SLATE is to replace the
venerable Scalable Linear Algebra PACKage (ScaLAPACK) library,
which has become the industry standard for dense linear algebra operations
in distributed-memory environments.
After two decades of operation,
ScaLAPACK is past the end of its life cycle and overdue for a replacement,
as it can hardly be retrofitted to support GPUs,
which are an integral part of today's HPC hardware infrastructure.

Primarily, SLATE aims to extract the full performance potential and maximum
scalability from modern HPC machines with large numbers nodes,
large numbers of cores per node, and multiple GPUs per node.
For typical dense linear algebra workloads, this means getting close
to the theoretical roofline peak performance and scaling to the full size of
the machine.
This is accomplished in a portable manner by relying on standards
such as MPI and OpenMP.
%
% SLATE functionalities will first be delivered to the ECP applications
% that most urgently require SLATE capabilities
% (NWChem, GAMESS, EXAALT, QMCPACK, CANDLE, etc.)
% and to other software libraries
% that rely on underlying dense linear algebra services
% (STRUMPACK, SuperLU, etc.).
Figure~\ref{fig:slate-architecture} shows the role of SLATE
in the ECP software stack.

While the initial objective of SLATE is to serve as a successful,
drop-in replacement for ScaLAPACK with support for GPU accelerators,
the ultimate goal of SLATE is to deliver dense linear algebra capabilities
beyond the capabilities of ScaLAPACK.
This includes new features such as communication-avoiding
algorithms and randomization algorithms, as well as the potential to
support variable size tiles and block low-rank compressed tiles.

\begin{figure}[htb]
    \centering
    \includegraphics[width=0.75\textwidth]{projects/2.3.3-MathLibs/2.3.3.13-CLOVER/SLATE-architecture.jpg}
    \caption{\label{fig:slate-architecture}
    SLATE in the ECP software stack.}
\end{figure}

\paragraph{Key  Challenges}

\begin{enumerate}

\item
\textbf{Designing from the ground up:}
The SLATE project's primary challenge stems from the need to design the package
from the ground up, as no existing software package offers
a viable path forward for efficient support of GPUs
in a distributed-memory environment.

\item
\textbf{Facing harsh hardware realities:}
SLATE is being developed in a difficult hardware environment, where virtually
all the processing power is on the GPU side.
Achieving efficiency requires aggressive offload to GPU accelerators
and careful optimization of multiple bottlenecks, including
interconnect technology lagging behind the computing
capabilities of the GPUs.

\item
\textbf{Facing harsh software realities:}
SLATE is being developed using cutting-edge software technologies,
and relies on modern C++ features and recent extensions
to the OpenMP standard, many of which are not fully supported by compilers
and their runtime environments.
In terms of GPU acceleration, standardized solutions are still in flux.
% Also, while complete parallel programming frameworks exist, at this stage
% they have to be considered research prototypes.

\end{enumerate}

\paragraph{Solution Strategy}

\begin{enumerate}

\item
\textbf{Evolving design:}
Due to the inherent challenges of designing a software package
from the ground up, the SLATE project started
with a careful analysis of the existing and emerging
implementation technologies~\cite{abdelfattah2017roadmap},
and followed with a phase
of laying out the initial design~\cite{kurzak2017designing}.
Since then, the team has rolled out new computational routines
and performance improvements quarterly.
While we continue to refactor as needed to achieve high performance, the basic
design has solidified and been published~\cite{gates2019slate-design}.

\item
\textbf{Focus on GPUs:}
Efficient GPU acceleration is the primary focus of performance
engineering efforts in SLATE.
Where applicable, highly optimized vendor implementations of GPU operations
are used, such as the batched \texttt{gemm} routine.
Where necessary, custom GPU kernels are developed, as in the case of computing
matrix norms.
Care is taken to hide communication by overlapping it with GPU computations.

\item
\textbf{Community engagement:}
The SLATE team interacts on a regular basis with the OpenMP community,
represented in ECP by the SOLLVE project, and with the MPI community,
represented in ECP by the OMPI-X project and the Exascale MPI project.
The SLATE team also engages the vendor community through our contacts
at Cray, IBM, Intel, NVIDIA, AMD, and ARM.

\end{enumerate}

\paragraph{Recent Progress}

During 2020, the SLATE team expanded the Hermitian eigenvalue solver to
the generalized Hermitian problem of the forms $Ax = \lambda Bx$, $ABx =
\lambda x$, or $BAx = \lambda x$. These are of strong interest in
mechanics and chemistry applications, among others.
%
We also implemented the polar decomposition with the QDWH algorithm,
%
wrote a Users' Guide and Developers' Guide to document the public and
internal APIs,
%
and added native C and Fortran 2003 APIs that access SLATE's native
matrix types from outside C++.
%
Further work was done on performance enhancements, with notable gains
for BLAS (gemm, herk), norms, Cholesky and QR factorizations.
%
A new build system using CMake and Spack has been developed, in
collaboration with the NWChemEx project to help include BLAS++ and
LAPACK++ in their project.
%
All developments are documented in SLATE Working Notes
\footnote{\url{http://www.icl.utk.edu/publications/series/swans}.}

\paragraph{Next Steps}

\begin{enumerate}

\item
\textbf{Port to AMD and Intel platforms:}
Originally, SLATE was developed using NVIDIA CUDA and cuBLAS. We are now
abstracting the backend to run on AMD and Intel platforms. BLAS++ will
serve as a portability layer, with calls to NVIDIA cuBLAS, AMD rocBLAS,
or Intel oneMKL, as appropriate for the platform. CUDA kernels will be
ported to a combination of HIP, SYCL, and OpenMP offload.

\item
\textbf{Optimizing QR, eigenvalue, and singular value routines:}
These are fundamental routines used by many projects.
We have observed these routines that are not performing as well as expected.
In some cases, such as eigenvalues, we have already identified
improvements to be made to the algorithm, such as refactoring loops to improve
parallelism. In other cases, we will analyze traces to identify and
correct problems.

\item
\textbf{Implementing divide-and-conquer algorithm:}
For singular value and Hermitian eigenvalue problems, the divide-and-conquer
algorithm exhibits better performance and parallel scalability than the
traditional QR-iteration based algorithm currently used in SLATE.

\item
\textbf{Non-symmetric eigenvalue problem:}
Computing the non-symmetric eigenvalue problem is significantly more
computationally expensive than the Hermitian eigenvalue problem. Our
implementation will leverage the latest advances, such as aggressive early
deflation, to achieve high performance.

\end{enumerate}
